% FortySecondsCV LaTeX template
% Copyright © 2019-2020 René Wirnata <rene.wirnata@pandascience.net>
% Licensed under the 3-Clause BSD License. See LICENSE file for details.
%
% Please visit https://github.com/PandaScience/FortySecondsCV for the most
% recent version! For bugs or feature requests, please open a new issue on
% github.
%
% Contributors
% ------------
% * ifokkema
% * Bertbk
% * Hespe
%
% Attributions
% ------------
% * fortysecondscv is based on the twentysecondcv class by Carmine Spagnuolo
%   (cspagnuolo@unisa.it), released under the MIT license and available under
%   https://github.com/spagnuolocarmine/TwentySecondsCurriculumVitae-LaTex
% * further attributions are indicated immediately before corresponding code


%-------------------------------------------------------------------------------
%                             ADDITIONAL PACKAGES
%-------------------------------------------------------------------------------
\documentclass[
	a4paper,
	% showframes,
	% vline=2.2em,
	% maincolor=cvgreen,
	% sidecolor=gray!50,
	% sectioncolor=red,
	% subsectioncolor=orange,
	% itemtextcolor=black!80,
	% sidebarwidth=0.4\paperwidth,
	% topbottommargin=0.03\paperheight,
	% leftrightmargin=20pt,
	% profilepicsize=4.5cm,
	% profilepicborderwidth=3.5pt,
	% profilepicstyle=profilecircle,
	% profilepiczoom=1.0,
	% profilepicxshift=0mm,
	% profilepicyshift=0mm,
	% profilepicrounding=1.0cm,
]{fortysecondscv}

% improve word spacing and hyphenation
\usepackage{microtype}
\usepackage{ragged2e}

% uncomment in case you don't want any hyphenation
% \usepackage[none]{hyphenat}

% take care of proper font encoding
\ifxetexorluatex
	\usepackage{fontspec}
	\defaultfontfeatures{Ligatures=TeX}
%	\newfontfamily\headingfont[Path = fonts/]{segoeuib.ttf} % local font
\else
	\usepackage[utf8]{inputenc}
	\usepackage[T1]{fontenc}
%	\usepackage[sfdefault]{noto} % use noto google font
\fi

% enable mathematical syntax for some symbols like \varnothing
\usepackage{amssymb}

% bubble diagram configuration
\usepackage{smartdiagram}
\smartdiagramset{
	% default font size is \large, so adjust to harmonize with sidebar layout
	bubble center node font = \footnotesize,
	bubble node font = \footnotesize,
	% default: 4cm/2.5cm; make minimum diameter relative to sidebar size
	bubble center node size = 0.4\sidebartextwidth,
	bubble node size = 0.25\sidebartextwidth,
	distance center/other bubbles = 1.5em,
	% set center bubble color
	bubble center node color = maincolor!70,
	% define the list of colors usable in the diagram
	set color list = {maincolor!10, maincolor!40,
	maincolor!20, maincolor!60, maincolor!35},
	% sets the opacity at which the bubbles are shown
	bubble fill opacity = 0.8,
}


%-------------------------------------------------------------------------------
%                            PERSONAL INFORMATION
%-------------------------------------------------------------------------------
%% mandatory information
% your name
\cvname{Zafeiropoulos \\ Haris}
% job title/career
\cvjobtitle{Bioinformatics, Systems\\[0.2em]  Biology, Microbial Ecology}

%% optional information
% profile picture
\cvprofilepic{pics/haris.jpeg}

% NOTE: ordering in sidebar will mimic the following order
% date of birth
\cvbirthday{April 24, 1991}
% short address/location, use \newline if more than 1 line is required
\cvaddress{Valestra 99, Heraklion, Greece}
% phone number
\cvphone{+30 694 909 3089}
% personal website
\cvsite{https://hariszaf.github.io/}
% email address
\cvmail{haris-zaf@hcmr.gr}
% pgp key
% \cvkey{4096R/FF00FF00}{0xAABBCCDDFF00FF00}
% % any other custom entry
% \cvcustomdata{\faFlag}{Chinese}

%-------------------------------------------------------------------------------
%                              SIDEBAR 1st PAGE
%-------------------------------------------------------------------------------
% add more profile sections to sidebar on first page
\addtofrontsidebar{
	% include gosquare national flags from https://github.com/gosquared/flags;
	% naming according to ISO 3166-1 alpha-2 country codes
	\graphicspath{{pics/flags/}}

	% social network accounts incl. proper hyperlinks
	\profilesection{Social Network}
		\begin{icontable}{2.5em}{1em}

			\social{\faGithub}
				{https://github.com/hariszaf}
				{Github Profile}
			\social{\faSkype}
				{haris.zaf2}
				{haris.zaf2}
			\social{\faTwitter}
				{https://twitter.com/haris_zaf}
				{@haris\_zaf}

		\end{icontable}

	\profilesection{Languages}
	\pointskill{\flag{GB.png}}{English}{5}
	\pointskill{\flag{FR.png}}{French}{3}


	\profilesection{Coding Skills}
		\pointskill{\faLinux}{Unix / Linux / AWK}{4}[5]
		\pointskill{}{Python 3.0}{4}[5]
		\pointskill{\flag{Matlab_Logo.png}}{Matlab}{3}[5]
		\pointskill{\textsf{R}}{R / Rstudio}{3}[5]
		\pointskill{\flag{cpp.png}}{C++}{3}[5]
		\pointskill{}{HTML / CSS / JS}{3}[5]
		\pointskill{}{Docker \& Singularity}{4}[5]
		\pointskill{\faCloudDownload*}{REST API}{4}[5]


	\profilesection{Math and Stats}
		\pointskill{\faTable}{Linear Algebra}{4}[5]
		\pointskill{\faCube}{Constraint-based \\    modelling}{3}[5]

}


%-------------------------------------------------------------------------------
%                              SIDEBAR 2nd PAGE
%-------------------------------------------------------------------------------
\addtobacksidebar{
	\profilesection{About Me}
	\aboutme{
            I am studying microbial ecology \& ecosystem functioning by exploiting knowledge aggregation \& data integration techniques.
            My interests focus on systems biology, metabolic modelling and microbial interactions.
            
	}

	\profilesection{My so-far research}
	\begin{sidebarminipage}
		\chartlabel{\href{http://pema.hcmr.gr/}{pema}}
	    \chartlabel{\href{http://prego.hcmr.gr/}{prego}}
		\chartlabel{\href{https://github.com/hariszaf/darn}{darn}}
		\chartlabel{\href{https://github.com/GeomScale/dingo}{dingo}}
	\end{sidebarminipage}


	\begin{figure}\centering
		\smartdiagram[bubble diagram]{
			\textcolor{white}{\textbf{microbial}} \\
			\textcolor{white}{\textbf{communities}}, 
			\textcolor{black!90}{metabarcoding},
			\textcolor{black!90}{flux sampling},
			\textcolor{black!90}{HPC},
			\textcolor{black!90}{omics},
			\textcolor{black!90}{data} \\
			\textcolor{black!90}{integration}
		}
	\end{figure}


	\profilesection{Barskills}
	\barskill{}{Coding/programming}{80} % \faLaptop
	\barskill{}{Scientific writing}{70} % \faBookReader
	\barskill{}{Microbial ecology}{60}  % \faVirus

	\profilesection{Memberships}
	\begin{memberships}
		\membership[4em]{pics/mkb.png}{\href{http://www.mikrobiokosmos.org/}{Mikrobiokosmos}}
		\membership[4em]{pics/isme.png}{\href{https://www.isme-microbes.org/}{International Society of Microbial Ecology}}

	\end{memberships}
}


\addtomoresidebar{
    \profilesection{My so-far research}
}


%-------------------------------------------------------------------------------
%                         TABLE ENTRIES RIGHT COLUMN
%-------------------------------------------------------------------------------
\begin{document}

\makefrontsidebar

\cvsection{Research projects - working Experience}
\begin{cvtable}[3]

	\cvitem{2022 - currently}{\href{https://3domics.eu}{3D'omics}}{Post-doc}
	{Generate 3D omics landscapes, achieving reconstructions of intestinal host microbiota ecosystems.}

	\cvitem{2022 - 2022}{\href{https://imbbc.hcmr.gr/project/eosc-life-gos/}{A workflow for marine Genomic Observatories data analysis}}{Research associate}
	{Making the large volumes of data produced by genomi observatories more easily interpretable by providing the taxonomic inventories of each sample in a timely manner and in a non-technical format}


	\cvitem{2018 - 2021}{\href{http://prego.hcmr.gr/}{PREGO: Process, environment, organism (PREGO)}}{PhD}
	{PREGO is a systems-biology approach to elucidate ecosystem function at the microbial dimension.}
	
	\cvitem{2019 -- present}{\href{elixir-greece.org}{ELIXIR–GR}}{technical support}
	{ELIXIR-GR is the Greek National Node of the ESFRI \href{https://elixir-europe.org/}{European RI ELIXIR}, a distributed e-Infrastructure aiming at the construction of a sustainable European infrastructure for biological information.}
	
	\cvitem{2018 - 2020}{\href{https://reconnect.hcmr.gr/}{RECONNECT}}{PhD}
	{RECONNECT is an Interreg V-B "Balkan-Mediterranean 2014-2020" project. It aims to develop strategies for sustainable management of Marine Protected Areas (MPAs) and Natura 2000 sites.}

\end{cvtable}


\cvsection{Education}

\cvsubsection{Graduate studies}
\begin{cvtable}[1.5]
	\cvitem{2018 -- 2022}{PhD in Bioinformatics }{\href{https://en.uoc.gr/}{University of Crete}, \href{https://www.biology.uoc.gr/en}{Biology department}}{}
	\cvitem{}{Dissertation}{\href{https://imbbc.hcmr.gr/}{IMBBC - HCMR}}
		{\href{https://www.openarchives.gr/aggregator-openarchives/edm/elocus/000018-dlib_5_7_d_metadata-dlib-1661248925-407543-18532.tkl?language=en}{Microbial communities through the lens of high throughput sequencing, data integration and metabolic networks analysis}}
	\\
	\cvitem{2016 -- 2018}{M.Sc. in  Bioinformatics}
	{\href{https://en.uoc.gr/}{University of Crete}, \href{http://www.english.med.uoc.gr/}{School of Medicine}}
		{grade: 9.1/10.0}
	\cvitem{}{Master Thesis}{\href{https://imbbc.hcmr.gr/}{IMBBC - HCMR}}
	    {\href{https://www.openarchives.gr/aggregator-openarchives/edm/elocus/000018-dlib_5_2_f_metadata-dlib-1545038085-364284-26948.tkl}{eDNA metabarcoding for biodiversity assessment: Algorithm design and bioinformatics analysis pipeline implementation}}

\end{cvtable}


\cvsubsection{Undergraduate studies}
	\begin{cvtable}[1.5]
		
	\cvitem{2011 -- 2016}{B.Sc. in Biology}{\href{https://en.uoa.gr/}{National \& Kapodistrian University of Athens} }
		{grade: 6.2 / 10.0}
	\cvitem{}{Bachelor Thesis}{\href{https://en.uoa.gr/schools_and_departments/school_of_science/}{School of Science}, \href{http://en.biol.uoa.gr/}{department of Biology}}
		{Morphology, morphometry and anatomy of species of the genus \textit{Pseudamnicola} in Greece}
		
\end{cvtable}


%  -----------------------------------------------------------------------
\newpage
\makemoresidebar
%  -----------------------------------------------------------------------


% ----------------------------------------------------------------------------
%                            PUBLICATIONS
% ----------------------------------------------------------------------------


\cvsection{Publications}
\begin{cvtable}

	\cvpubitem{Metabolic models of human gut microbiota: Advances and challenges.}
	{Garza, D.R., Gonze, D., \textbf{Zafeiropoulos, H.}, Liu, B. and Faust, K.}
	{\textit{Cell Systems}, 14(2), pp.109-121., DOI: \href{https://doi.org/10.1016/j.cels.2022.11.002}{10.1016/j.cels.2022.11.002}}
	{2023}

    \cvpubitem{Deciphering the community structure and the functional potential of a hypersaline marsh microbial mat community.}
	{Pavloudi, C. and \textbf{Zafeiropoulos, H.} }
	{\textit{FEMS Microbiology Ecology} 98.12 (2022): fiac141. DOI: \href{https://doi.org/10.1093/femsec/fiac141}{10.1093/femsec/fiac141}}
	{2022} \\

	\cvpubitem{Automating the Curation Process of Historical Literature on Marine Biodiversity Using Text Mining: The DECO Workflow.}
	{Paragkamian, S., Sarafidou, G., Mavraki, D., Pavloudi, C., Beja, J., Eliezer, M., Lipizer, M., Boicenco, L., Vandepitte, L., Perez-Perez, R., \textbf{Zafeiropoulos, H.} and others}
	{\textit{Frontiers in Marine Science}, 9, p.940844., DOI: \href{https://doi.org/10.3389/fmars.2022.940844}{10.3389/fmars.2022.940844}}
	{2022} \\

    \cvpubitem{PREGO: A Literature and Data-Mining Resource to Associate Microorganisms, Biological Processes, and Environment Types}
	{\textbf{Zafeiropoulos, H.,} Paragkamian, S., Ninidakis, S., Pavlopoulos, G., A., Jensen, L., J., Pafilis, E. }
	{\textit{Microorganisms} 10, no. 2 (2022): 293, DOI: \href{10.3390/microorganisms10020293}{10.3390/microorganisms10020293}}
	{2022} \\

	\cvpubitem{The Dark mAtteR iNvestigator (DARN) tool: getting to know the known unknowns in COI amplicon data.}
	{\textbf{Zafeiropoulos, H.,} Gargan, L., Hintikka, S., Pavloudi, C. \& Carlsson, J.}
	{\textit{Metabarcoding and Metagenomics}, 5, p.e69657., DOI:\href{https://doi.org/10.3897/mbmg.5.69657}{10.3897/mbmg.5.69657}}
	{2021} \\

	\cvpubitem{0s \& 1s in marine molecular research: a regional HPC perspective}
	{\textbf{Zafeiropoulos, H.,} Gioti A., Ninidakis S., Potirakis A., ..., \& Pafilis E.}
	{\textit{GigaScience}, 9(3), p.giab053, DOI: \href{https://doi.org/10.1093/gigascience/giab053}{10.1093/gigascience/giab053}}
	{2021} \\

    \cvpubitem{Geometric Algorithms for Sampling the Flux Space of Metabolic Networks}
	{Chalkis, A., Fisikopoulos, V., Tsigaridas, E. \& \textbf{Zafeiropoulos, H.}}
	{\textit{37th International Symposium on Computational Geometry (SoCG 2021)}, 21:1--21:16, 189, DOI: \href{https://doi.org/10.4230/LIPIcs.SoCG.2021.21}{10.4230/LIPIcs.SoCG.2021.21}}
	{2021} \\

	\cvpubitem{The Santorini Volcanic Complex as a Valuable Source of Enzymes for Bioenergy}
	{Polymenakou, P.N., Nomikou, P., \textbf{Zafeiropoulos, H.}, Mandalakis, M., Anastasiou, T.I., Kilias, S., Kyrpides, N.C., Kotoulas, G. \& Magoulas,A.}
	{\textit{Energies}, 14(5), p.1414.}{2021} \\

	\cvpubitem{PEMA: a flexible Pipeline for Environmental DNA Metabarcoding Analysis of the 16S/18S ribosomal RNA, ITS \& COI marker genes}{\textbf{Zafeiropoulos, H.}, Viet, H.Q., Vasileiadou, K., Potirakis, A., Arvanitidis, C., Topalis, P., Pavloudi, C. \& Pafilis, E}
	{\textit{GigaScience}, 9(3), p.giaa022, DOI:\href{https://doi.org/10.1093/gigascience/giaa022}{10.1093/gigascience/giaa022}}{2020}
	
\end{cvtable}




%  -----------------------------------------------------------------------
\newpage
\makemoresidebar
%  -----------------------------------------------------------------------

% ------------------------------------------------------------------------
%                            PRESENTATIONS
% ------------------------------------------------------------------------


\cvsection{Conferences}
\begin{cvtable}


	

	\cvitem{2022}{1st Applied HoloGenomics conference @ Bilbao, Spain}
    	{\href{https://appliedhologenomicsconference.eu}{AHC2022}}
	    {\textit{Poster presentation} \\ \texttt{microbetag}: a microbial co-occurrence network annotator}{}

    \\

	\cvitem{2022}{18th International Symposium on Microbial Ecology @ Laussane, Switzerland}
    	{\href{https://isme18.isme-microbes.org}{ISME18}}
	    {\textit{Poster presentation} \\ Reverse ecology and systems biology approaches to identify key processes ensuring life in extreme environments}{}

    \\


	\cvitem{2021}{Bioinformatics Open Source Conference \\ (BOSC) - online}
    	{\href{https://www.open-bio.org/events/bosc-2021/}{BOSC2021}}
	    {\textit{Flash talk} \\ \texttt{dingo}: A python library for metabolic networks sampling \& analysis.)}{}

    \\
    
	\cvitem{2021}{1st DNAQUA International Conference - online}
    	{\href{https://symposium.inrae.fr/dnaqua-conference-evian2021/}{DNAQUA}}
	    {\textit{Flash talk} \\ PEMA v2: addressing metabarcoding bioinformatics analysis challenges}{}
	    
	\\

	\cvitem{2020}{Federation of European Microbiological \\ Societies (FEMS) - online}
    	{\href{https://fems2020belgrade.com/}{FEMS2020}}
	    {\textit{Flash talk} \\ Mining literature and -omics (meta)data to associate microorganisms, biological processes and environment types}{}

    \\

	\cvitem{2020}{PyData Global - online}
    	{\href{https://pydata.org/global2020/}{PyData2020}}
	    {\textit{Oral presentation} \\ \href{https://www.youtube.com/watch?v=zg8KFZ_LbHM&t=1s}{Geometric and statistical methods in systems biology: the case of metabolic networks}}{}

    \\

    \cvitem{2019}{network2019: 4th Symposium on Ecological Networks @ Paris, France}
    	{\href{https://network2019.sciencesconf.org}{network2019}}
    	{Participation.}{}

	\\
    
    \cvitem{2019}{8th International Barcode of Life Conference @ Trondheim, Norway}
    	{\href{http://dnabarcodes2019.org/}{iBOL}}
    	{\textit{ePoster presentation} \\ \href{}{P.E.M.A.: a Pipeline for Environmental DNA Metabarcoding Analysis}}{}

	\\

	\cvitem{2018}{European Conference on Computational Biology (ECCB) 2018 @ Athens, Greece}
    	{\href{https://www.iscb.org/cms_addon/events/details.php?uid=2508}{ECCB18}}
    	{\textit{Poster presentation} \\ \href{}{P.E.M.A.: a Pipeline for Environmental DNA Metabarcoding Analysis}}{}



\end{cvtable}






\cvsection{Workshops}
\begin{cvtable}

	\cvitem{2022}{Microbial communities: current approaches and open challenges}
    	{\href{https://www.newton.ac.uk/event/umcw06/}{UMCW06}}
    	{Participant}{}

	\cvitem{2021}{Microbiome Data Analyses Workshop Online}
    	{\href{https://meetinghand.com/e/mdawo}{MDAWO}}
    	{\textbf{Tutorial:} PEMA: a flexible Pipeline for Environmental DNA Metabarcoding Analysis of the 16S/18S rRNA, ITS and COI marker genes.}{}

	

\end{cvtable}
%  -----------------------------------------------------------------------
\newpage
\makebacksidebar
%  -----------------------------------------------------------------------


\cvsection{Teaching - Supervision}
\begin{cvtable}

	\cvitem{2023}{"Development of a Cytoscape App for microbe-microbe association networks"}
	{\href{https://summerofcode.withgoogle.com/programs/2023/projects/7IKSInMx}{post}}
	{Google Summer of Code mentor}{}\\

	\cvitem{2023}{eDNA metabarcoding, pipeline development \& high performance computing}
	{\href{https://docs.google.com/presentation/d/1FSObXaOt0HFq30UBypO4ui13oy1cbflxLRQJPQwfjC4/edit?usp=sharing}{slides}}
	{Lecture on MSc "Applied Bioinformatics \& Data Analysis" in [\href{https://bioinfo.mbg.duth.gr}{Democritus University of Thrace}, Greece}{}\\

	\cvitem{2022}{Genome-scale model reconstruction}
	{\href{https://colab.research.google.com/drive/1uL-oiSNAQoAL1qWyS4z_y94kWhHyT4hA?usp=sharing}{GColab notebook}}
	{Lectures on "Master en bioinformatique et modélisation" of the~\href{https://www.ulb.be/fr/programme/ma-binf}{Université Libre de Bruxelles}.}{}\\

	\cvitem{2020}{Environmental DNA \& DNA metabarcoding}
	{\href{https://docs.google.com/presentation/d/19TVFO3Rw-1owvrqVdzmUzQUrYHZ1Vaw4ctVjBso-n-E/edit?usp=sharing}{slides}}
	{Lecture on MSc "Environmenta Biology" in~\href{http://envbio.biology.uoc.gr}{University of Crete}.}{}


	% \cvitem


\end{cvtable}



\cvsection{Awards}
\begin{cvtable}

	\cvitem{2021}{Google Summer of Code}
	{\href{https://summerofcode.withgoogle.com/}{GSoC}}
	{Project title: \textit{\href{https://hariszaf.github.io/gsoc2021/}
	{From DNA sequences to metabolic interactions: building a pipeline to extract key metabolic processes}}}{}

    \\

	\cvitem{2021}{European Molecular Biology Organization Short-Term Fellowship}
	{\href{https://www.embo.org/}{EMBO}}
	{Project title: \textit{"Exploiting data integration, text-mining and computational geometry to enhance microbial interactions inference from  co-occurrence networks"}}{}

    \\

	\cvitem{2020}{Federation of European Microbiological Societies \\ Meeting Attendance Grant}
	{\href{https://fems-microbiology.org/}{FEMS}}
	{for joining the \textit{"Metagenomics, Metatranscript- omics and \\ 
	multi ’omics for microbial community studies"} Physalia course}{}

    \\
    
	\cvitem{2019}{Short Term Scientific Mission (STSM) \\ @ DNAqua-net COST action}
	{\href{http://dnaqua.net/}{DNAqua-net}}
	{Project title \textit{A comparison of bioinformatic pipelines and sampling techniques to enable benchmarking of DNA metabarcoding}. \href{http://dnaqua.net/wp-content/uploads/2019/08/Zafeiropoulos.pdf}{Report.}}{}


    \\
	
	\cvitem{2018}{Best Poster Award @ Hellenic Bioinformatics conference}
	{\href{https://hscbio.wordpress.com/}{HBIO}}
	{for \textit{PEMA: a Pipeline for Environmental DNA Metabarcoding Analysis}}{}
	
\end{cvtable}


\cvsection{References}
\begin{cvtable}

	\cvitemshort{\href{http://lab42open.hcmr.gr/people/evangelospafilis/}{Pafilis Evangelos}}
	    {Institute of Marine Biology, Biotechnology \& Aquaculture, Hellenic Centre for Marine Research, (IMBBC - HCMR
	   % {\href{https://imbbc.hcmr.gr/}{IMBBC HCMR}}
	    ),
	    Gournes, Pediados, P.O. Box 2214, Heraklion Crete 71003, Greece \\ \faPhone: +30 2810 337740 | 
	    \faAt: \href{mailto:pafilis@hcmr.gr}{pafilis@hcmr.gr}}
	    
	\\
	  
	\cvitemshort{\href{https://ladoukakis.weebly.com/}{Emmanuel D. Ladoukakis}}
	    {Biology Department, University of Crete,
	    Voutes University Campus, Iraklio, GR 70013 Biology building, 3rd floor, Office: 316a 
	    \\ 
	    \faPhone: +30 2810394067 | 
	    \faAt: \href{mailto:ladoukakis@biology.uoc.gr}{ladoukakis@biology.uoc.gr}}
	  
	\\ 

    
	\cvitemshort{\href{https://who.paris.inria.fr/Elias.Tsigaridas/}{Elias Tsigaridas}}
	    {Institut de Mathématiques de Jussieu - Paris Rive Gauche, Sorbonne Université - Campus Pierre et Marie Curie, Case courrier 247, 4, place Jussieu, 75252, Paris Cedex 05, France \\ 
	    \faPhone: +33 014427 8539 | \faAt: \href{mailto: elias.tsigaridas@inria.fr}{elias.tsigaridas@inria.fr}}

    \\

	\cvitemshort{\href{http://computational-genomics.weebly.com/}{Christoforos Nikolaou}}
	    {Department of Biology, University of Crete, Gennimata 34,71305, Heraklion, Crete, Greece \\ 
	    \faAt: \href{mailto:cnikolaou@fleming.gr}{cnikolaou@fleming.gr}}


\end{cvtable}





%  -----------------------------------------------------------------------
% \newpage
%  -----------------------------------------------------------------------


\end{document}
